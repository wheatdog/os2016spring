
\documentclass[12pt, a4paper]{article}
\usepackage{xeCJK}
%\setCJKmainfont{WenQuanYi Micro Hei Light}
\setCJKmainfont{Noto Sans CJK TC Light}
\XeTeXlinebreaklocale "zh"
\XeTeXlinebreakskip = 0pt plus 1pt
\usepackage{amsmath}
\usepackage[top=1in, bottom=1in, left=1in, right=1in]{geometry}

\title{Report of Operating System Project 1}
\author{戴佑全\\黃子賢\\劉彥廷}

\begin{document}
\maketitle

\section{Design of Child}
schedule policy設為SCHED\_FIFO主程式main一開始將自己的priority設為98。
main要叫一個child跑時,會透過pipe告訴child要跑幾個time unit,再將此child的priority設為99
child跑了指定的時間後 會將自己的priority設為97 把主控權還給main。

\section{FIFO}
\subsection{Design}

資料結構 : 一開始打算以link list來當作ready queue,但後來想到有給process的數量再
加上並沒有用到link list可以在中間隨意插入的優點,所以改用array當作ready queue,
並有2個變數紀錄現在執行的child與下一個要被fork的child。

FIFO是一個很直觀的 scheduler,因為只有一顆CPU,所以我的作法是parent將
child要跑多少時間跟child說,之後將child的priority提到最高,而 parent
會重新拿到CPU有2種情況,

\begin{enumerate}
  \item 因為可能中途要fork其他的child,所以child需要在那時將CPU還給parent,這裡 使用的方式是將parent的priority設為98,換child時將child提成99,當child要回給 parent時將自己設為97,就能完成context switch了,而因為此時child還沒執行完,只是暫時換parent,所以在parent中現在執行的變數不變,所以fork完之後會接續做。
  \item 當child執行完後,輪到parent時,將parent中現在執行的變數增加並wait child 以防有zombie。 所以parent的流程是,先檢查是否在idle(現在執行的child = 下一個要被 fork的child),如果是則做for loop直到下一個child被出來,再來是檢查需不需要fork, 因為並非每次輪到parent時都要fork,再來是將child需要跑的時間告訴child,時間為 min(下一個要被fork的child的 ready time, 現在執行的child的execution time)。
\end{enumerate}

\subsection{Result}

\begin{enumerate}
\item \texttt{FIFO\_2.txt}
\begin{verbatim}
P1 3515
P2 3516
P3 3517
P4 3518

[Project1] 3515 1461500526.987120466 1461500691.939705154
[Project1] 3516 1461500691.939706083 1461500702.258074897
[Project1] 3517 1461500702.258075945 1461500704.295347989
[Project1] 3518 1461500704.295349110 1461500706.332647694
\end{verbatim}
\end{enumerate}

\subsection{Comparison}
與理想中的差不多,但是在dmesg稍微有點誤差,原因大概是parent 與child之間context
switch造成的,除了priority有保護順序外,因為FIFO並不考慮preemptive,所以只有上一
個做完下一個才會做,再加上每次child死亡都會wait所以這也會保護到答案順序的正確。

\section{SJF}

\subsection{Design}
設計跟FIFO差別不大,唯一的差別是在fork完之後要做一次sort,有2種情況,
\begin{enumerate}
\item 現在有child正在執行中,那麼只sort除了這支child的其他已被fork出來的child。
\item 現在沒有child正在執行中,那麼sort所有已被fork出來的child。除此之外剩下皆與FIFO相同。
\end{enumerate}

\subsection{Result}
\begin{enumerate}
\item \texttt{SJF\_3.txt}
\begin{verbatim}
P1 2730
P4 2733
P5 2734
P6 2735
P7 2736
P2 2731
P3 2732
P8 2737

[Project1] 2730 1461506590.620579045 1461506597.056205686
[Project1] 2733 1461506597.056208329 1461506597.077219594
[Project1] 2734 1461506597.077221749 1461506597.098146308
[Project1] 2735 1461506597.098147463 1461506605.898578152
[Project1] 2736 1461506605.898581016 1461506614.651940234
[Project1] 2731 1461506614.651943364 1461506625.677029525
[Project1] 2732 1461506625.677032127 1461506641.046276916
[Project1] 2737 1461506641.046279188 1461506660.897146737
\end{verbatim}
\end{enumerate}

\subsection{Comparison}
SJF與FIFO很像,都是不preemptive,所以同FIFO,順序並不太會錯誤,在來就是因為要頻繁的sort所以在時間誤差方面會比較大。

\section{PSJF}
\subsection{Design}
\begin{enumerate}
\item Data structure: A poor man's priority queue, implemented by an array and
  qsort. The smaller one's exction time is, the higher its priority is.
  \begin{enumerate}
  \item Insert: $O(n \log n)$
  \item Pull: $O(1)$
  \end{enumerate}
\item Main while loop:
\begin{verbatim}
while(there exists non-finished processes) do
  if(clock is the next ready process's ready time) do
    fork this process
    insert this process to priority queue
  else
    wait until next ready process 
    continue
  end

  pull the process from the priority queue
  if(next ready process's ready time is smaller than this process's finish time) do
    this process can only run (next ready process's ready time - clock)
  else
    this process can run til it finishs
    wait this child, and mark this process as finished
  end
end
\end{verbatim}
\end{enumerate}

\subsection{Result}
\begin{enumerate}
\item \texttt{PSJF\_2.txt}
\begin{verbatim}
P2 3195
P1 3194
P4 3197
P5 3198
P3 3196

[Project1] 3195 1461556744.671319569 1461556746.818243380
[Project1] 3194 1461556742.541598901 1461556751.107887218
[Project1] 3197 1461556753.257726208 1461556757.596873203
[Project1] 3198 1461556757.596918243 1461556759.755637682
[Project1] 3196 1461556751.107896674 1461556766.169973474
\end{verbatim}
\end{enumerate}

\subsection{Comparison}
Basicly, the result is as same as the prediction. Sometimes, the error will be up
to 0.2 second. The reason might be that operation system did some content switch
to run other applications.

\section{RR}

\subsection{Design}
一開始  用\texttt{array}存下所有\texttt{process}資訊並照\texttt{ready time}排序。
\texttt{ready queue}用\texttt{circular linked list}實作,指標變數\texttt{cur\_job}指著\texttt{circular linked list}中正要執行的\texttt{process}(也就是\texttt{ready queue}的第一個\texttt{process}。)
\texttt{cur\_job->next}是\texttt{ready queue}中下一個要執行的\texttt{process},\texttt{cur\_job->last}是\texttt{ready queue}中最後一個\texttt{process}。
當目前的\texttt{process}的時間用完,要換下一個時,只要\texttt{cur\_job = cur\_job->next},目前的\texttt{process}就排到\texttt{ready queue}中的最後。
當新的\texttt{process ready},就要排進\texttt{ready queue}的最後,也就是將它塞進\texttt{cur\_job->last}。

\texttt{while loop}每個\texttt{iteration}主要流程:
一開始會檢查是否有\texttt{process ready},有就\texttt{fork child}。
接著如果\texttt{ready queue}是空的\texttt{(cur\_job == NULL)}: 
  就計算要等多久下個\texttt{process}會\texttt{ready},等完後進入下一個\texttt{iteration}。
否則:
  決定這次\texttt{cur\_job}要跑多久,跑完後做對應的操作,進入下一個\texttt{iteration}。
所有\texttt{process}執行完畢後,跳出\texttt{while loop},程式結束。

每個\texttt{iteration}中,決定\texttt{cur\_job}要跑多久的判斷和對應的操作:
比較
\begin{enumerate}
\item \texttt{cur\_job}這次\texttt{time quantum}剩餘的時間\texttt{(remain)}
\item 下個\texttt{process ready}的時間\texttt{(p\_arr[next].r\_time - cur\_time)}
\item \texttt{cur\_job} 剩餘的執行時間\texttt{(cur\_job->process->remain\_time)}
\end{enumerate}

時間最短的就是這個\texttt{iteration}中\texttt{cur\_job}要跑的時間
如果選出來的時間是1,跑完後\texttt{cur\_job}要換下個\texttt{process}(也就是\texttt{cur\_job->next})
如果是2,跑完後進入下個\texttt{iteration}就會\texttt{fork child}
如果是3,完後\texttt{cur\_job}要換下個\texttt{process}且移除目前跑完的\texttt{process}

\subsection{Result}
\begin{verbatim}
P3 2653
P1 2651
P2 2652
P6 2656
P5 2655
P4 2654

[Project1] 2653 1461682248.715317957 1461682276.941976396
[Project1] 2651 1461682242.580521729 1461682278.953463652
[Project1] 2652 1461682245.679939291 1461682280.969540910
[Project1] 2656 1461682256.829051535 1461682297.064796458
[Project1] 2655 1461682254.806290238 1461682301.062410827
[Project1] 2654 1461682252.791221494 1461682303.065176403
\end{verbatim}
\subsection{Comparison}
結果大致符合預期,但相距10000 time unit左右可以觀察到0.2秒的誤差, 可能是main要處理
children的context swith 時產生的overhead所導致的。
\section{Contribution of Each Member}
\begin{description}
\item[戴佑全:] system call, RR
\item[黃子賢:] child, FIFO, SJF
\item[劉彥廷:] main, PSJF
\end{description}

\end{document}​